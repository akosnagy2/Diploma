%----------------------------------------------------------------------------
\chapter{P�lyatervez�s elm�lete - 5 oldal CsG/NA}
%----------------------------------------------------------------------------

%----------------------------------------------------------------------------
\section{Megoldand� feladat}
%----------------------------------------------------------------------------


%----------------------------------------------------------------------------
\section{Lok�lis m�dszerek}
%----------------------------------------------------------------------------
A topol�giai felt�tel defin�ci�ja:

%TODO a megfelel� heyleken k�ne space, illetve valahogy align egym�shoz k�pset
\begin{align}\label{eq:topologicalProp}
\forall \epsilon > 0, \exists \eta > 0, \forall q_{0},q_{1} \in C \\ \notag
d_{\mathcal{C}}(q_{0},q_{1}) < \eta \rightarrow d_{\mathcal{C}}\left(q_{0}, Steer(q_{0},q_{1})(\sigma)\right) < \epsilon \\ \notag
\forall \sigma \in [q,S]
\end{align}

%----------------------------------------------------------------------------
\section{Glob�lis m�dszerek}
%----------------------------------------------------------------------------
